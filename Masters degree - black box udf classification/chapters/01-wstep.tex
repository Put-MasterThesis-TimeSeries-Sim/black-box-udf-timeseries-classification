
\chapter{Wstęp}

\subsection{Anna - zrozumienie pracy}
The goal of the thesis is to compare couple of algorithms, that are used to analyse time series.
Data, that will be used during experiments comes from real experiments and shows consumption of CPU and RAM during execution of specific functions. 
\par
The main task is to find out, which algorithms will be the most effective in determining which function was executed based on given time series. To do so, we need to take into account how the time series looks like – for example if it has a lot of outliers, or if the scale may differ a lot between measurements. That’s because there are some algorithms that are best for particular goals but no good for others. After choosing algorithm it is often needed to fit the algorithm’s
parameters for distinct task and data.
\par 
The algorithms are usually based on some observations and assumptions, so probably they could be divided into groups, that were based on the same grounds, or have common origin. In this thesis differences and similarities of different algorithms for comparing time series will also be analysed.
The general plan for working on this thesis could look like this:
\begin{enumerate}
\item Research of different algorithms for Time Series Analysis
\item Choosing couple of algorithms for deeper analysis
\item Comparing chosen algorithms in terms of common idea and origin. Getting to know chosen algorithms as best as possible
\item Experimenting and adjusting algorithms for the task
\item Comparing effectiveness of chosen algorithms
\end{enumerate}
\newpage
\subsection{Bartosz - zrozumienie pracy}
User-Defined Functions (UDF in short) are widely used in data processing pipelines. They are custom functions created in one of the languages that are available in the environment (for example scala or SQL). They are often seen as a ‘black box’, meaning that they can be run by supplying input parameters without the knowledge of what’s exactly happening during the processing. 

The goal of this thesis is to use runtime characteristics (RAM and CPU usage) of the running UDFs and based on them classify them into groups using machine learning or statistic methods.
Since the data is in time-series format, similarity measures for that type would be used based on available python libraries and then fed into a machine learning algorithm that can classify the UDF based on those measures.

In this work, we will be checking different algorithms and comparing them with each other to find the best method for this use case.



